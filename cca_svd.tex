%-*- mode: LaTex; outline-regexp: "\\\\section\\|\\\\subsection";fill-column: 80; -*-
\documentclass[12pt]{article}
\usepackage[longnamesfirst]{natbib}
\usepackage[usenames]{color}
\usepackage{graphicx}  % Macintosh pdf files for figures
\usepackage{amssymb}   % Real number symbol {\Bbb R}
\input{../../standard}

% --- margins
\usepackage{../sty/simplemargins}
\setleftmargin{1in}   % 1 inch is NSF legal minimum
\setrightmargin{1in}  % 1 inch is NSF legal minimum
\settopmargin{1in}    % 1 inch is NSF legal minimum
\setbottommargin{1in} % 1 inch is NSF legal minimum

% --- Paragraph split, indents
\setlength{\parskip}{0.00in}
\setlength{\parindent}{0in}

% --- Line spacing
\renewcommand{\baselinestretch}{1.5}

% --- page numbers
\pagestyle{empty}  % so no page numbers

% --- Hypthenation
\sloppy  % fewer hyphenated
\hyphenation{stan-dard}
\hyphenation{among}

% --- Customized commands, abbreviations
\newcommand{\TIT}{{\it  {\tiny CCA SVD notes (\today)}}}

% --- Header
\pagestyle{myheadings}
\markright{\TIT}

% --- Title

\title{ Notes on CCA and SVD in Text }
\date{\today}

%%%%%%%%%%%%%%%%%%%%%%%%%%%%%%%%%%%%%%%

\begin{document}
% \maketitle 



%--------------------------------------------------------------------------
\section{Modeling paradigm, notation}
\label{sec:paradigm}
%--------------------------------------------------------------------------

Need to settle notation, terminology: document length, number tokens, document
name.


Our perspective mimics the familiar convention that statistics concerns the
analysis of variables measured on observations, and we can view those
observations as realizations of a stochastic process or sampled data from a
population.  For analyzing text, {\em tokens} form the observations, and
variables are constructed from the analysis of how tokens {\em collocate} within
a variety of {\em contexts}.  Examples of contexts include documents, word types
before and after, or even just adjacency.

The paradigm we adopt follows these unsupervised steps:
\begin{enumerate}
 \item
 Compute the token $\times$ type matrix $W$.  This is a large, very sparse matrix with
 elements $W_{ij} \in \{0,1\}$.  $W$ has $N$ rows, where $N$ is the length of
 the source corpus, and $M$ columns, where $M$ is the size of the vocabulary
 (the number of types, which includes OOV and EOL markers). $W_{ij}=1$ implying
 that the $i$th word in the source corpus is of type $j$.
 \item
 Define the context matrix $C$.  $C$ defines a context for the sequence of
 tokens, so $C$ has $N$ rows as well.  Examples of the context are
  \begin{enumerate}
    \item Documents, where $C$ has the form resembling a Kronecker matrix,
                $C = I \otimes I_{N_i}$ 
          where $N_i$ is the length of the $i$th document.
    \item $W_1$, the following word (as used to form a bigram).
  \end{enumerate}
 \item
 Compute the SVD of $W'C = UDV'$, perhaps with some additional normalization or
 scaling that produces an analysis closer to what is obtained with a CCA (see
 further notes in section \ref{sec:ccasvd}.  The columns of $U$ define the
 location of {\em eigentypes}.
 \item
 Form regressors as centroids (perhaps weighted centroids) of the eigentype
 coordinates for the types that make up each document.
\end{enumerate}

%--------------------------------------------------------------------------
\section{CCA, via Lagrange multipliers}
\label{sec:ccalag}
%--------------------------------------------------------------------------

 The problem is to maximize
 \begin{equation}
    \max_{a,b} \rho(a,b)=
               \frac{a'S_{12}b}
                    {\left((a'S_{11}a)(b'S_{22}b)\right)^{1/2}}
 \label{eq:cancor}
 \end{equation}
 Since it is scale invariant, assume $a'S_{11}a = b'S_{22}b = 1$, and
reformulate with Lagrange multipliers as
 \begin{equation}
    \max_{a,b} a'S_{12}b - \la_1(a'S_{11}a-1) - \la_2 (b'S_{22}b-1)
 \label{eq:cca}
 \end{equation}
 Taking vector derivatives with respect to $a$ and $b$ gives the system of
 equations 
 \begin{eqnarray*}
   -2\la_2 S_{11} a + S_{12} b &=& 0  \cr
   S_{21} a - 2\la_2 S_{22} b &=& 0
\end{eqnarray*}
 Multiply the first equation by $a$ and use the constraint $a'S_{11}a = 1$ and multiply the second by $b$ and you see that $ 2 \la_1 = 2 \la_2 = \rho(a,b) \equiv \la$.  Now collect the two equations into matrix form as
 \begin{displaymath}
   \left( \begin{array}{cc}
     -\la S_{11} & S_{12} \cr
            S_{21} & -\la S_{22} 
   \end{array} \right)
   \left( \begin{array}{c}  a \cr b \end{array} \right)
   = 
   \left( \begin{array}{c} 0 \cr 0 \end{array} \right)
 \end{displaymath}
 For there to be a nontrivial solution, the matrix must be singular, with
determinant 0. The determinant being zero allows us to solve for $\la$ by using the expression for the determinant of a partitioned matrix:
 \begin{displaymath}
    |-\la  S_{11}| \; |-\la S_{22} - S_{21}S_{11}^{-1}S_{12}| \;
    = 
    |-S_{11}| \; |S_{22}| \;|\la^2I - S_{22}^{-1}S_{21}S_{11}^{-1}S_{12}| \; = 0
 \end{displaymath}
 Canceling the leading terms (product is zero) means that $\la^2$ is an eigenvalue of 
  $S_{22}^{-1}S_{21}S_{11}^{-1}S_{12}$. 

 Now that we know the values of the Lagrange multipliers (and the value of 
the canonical correlation $\rho(a,b)$), we can return to the
system of equations.  From the equation $d/da = 0$, we get
 \begin{displaymath}
    a = \frac{1}{\la} S_{11}^{-1} S_{12} b   
 \end{displaymath}
 Plug this into the second equation and you obtain
 \begin{displaymath}
   \frac{1}{\la} S_{21}S_{11}^{-1}S_{12} b - \la S_{22} b = 0
 \end{displaymath}
 which then implies that $b$ is an eigenvector with eigenvalue $\la^2$
 \begin{displaymath}
   (S_{22}^{-1} S_{21} S_{11}^{-1} S_{12}) \, b \; = \; \la^2 \, b \;.
 \end{displaymath}
 You get a similar expression for $a$ (just swap the 1s and 2s).


%--------------------------------------------------------------------------
\section{CCA, via Cauchy-Schwarz}
\label{sec:ccacs}
%--------------------------------------------------------------------------

The problem as given above in \eqn{eq:cancor} is 
 \begin{equation}
    \max_{a,b} \rho(a,b)=
               \frac{a'S_{12}b}
                    {\left((a'S_{11}a)(b'S_{22}b)\right)^{1/2}}
 \end{equation}
Rather than take the calculus route, this time use C-S to get the maximum
value.  Start by ``standardizing'' the coordinates, expressing $\rho(a,b)$ in
terms of 
 \begin{displaymath}
     u = S_{11}^{1/2} a \quad \mbox{and} \quad  v = S_{22}^{1/2}
 \end{displaymath}
so that we can write $\rho(a,b)$ as
 \begin{displaymath}
    \max_{u,v} \frac{u'S_{11}^{-1/2} S_{12} S_{22}^{-1/2}\,v}
                    {\left((u'u)(v'v)\right)^{1/2}}      
 \end{displaymath}
Now apply C-S to the {\em square} of the ratio,
 \begin{displaymath}
   \frac{((u'S_{11}^{-1/2} S_{12} S_{22}^{-1/2}\,)v)^2}
                    {\left((u'u)(v'v)\right)}
   \le 
   \frac{
    (u'S_{11}^{-1/2}S_{12}S_{22}^{-1/2}S_{22}^{-1/2}S_{21}S_{11}^{-1/2}u)
    (v'v)}
     {(u'u) (v'v) }
   = 
   \frac{
    (u'S_{11}^{-1/2}S_{12}S_{22}^{-1}S_{21}S_{11}^{-1/2}u)}
     {(u'u)}
 \end{displaymath}
 This is precisely an eigenvalue problem, implying that 
 \begin{displaymath}
    u\mbox{ is an eigenvector of }
        M = S_{11}^{-1/2}S_{12}S_{22}^{-1}S_{21}S_{11}^{-1/2}
 \end{displaymath}
 That's not the matrix that we want, but we can get the eigenvector for the CCA
 from this solution easily. Because $u$ is an eigenvector of $M$
 \begin{displaymath}
     M u = \la u \;.
 \end{displaymath}
 If we go back to the original coordinates, $u = S_{11}^{1/2}a$, then we get
 that $a$ is an eigenvector (and the eigenvalue does not change):
 \begin{displaymath}
    M S_{11}^{1/2} a = \la S_{11}^{1/2}  \quad \Rightarrow \quad
    (S_{11}^{-1}S_{12}S_{22}^{-1}S_{21}) a = \la a
 \end{displaymath}
 This all agrees with the Lagrange results in Section \ref{sec:ccalag}.
%--------------------------------------------------------------------------
% References
%--------------------------------------------------------------------------

\bibliography{../../../references/stat,../../../references/TextPapers/text}
\bibliographystyle{../bst/ims}

\end{document} %==========================================================
